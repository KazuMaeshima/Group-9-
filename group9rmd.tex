% Options for packages loaded elsewhere
\PassOptionsToPackage{unicode}{hyperref}
\PassOptionsToPackage{hyphens}{url}
%
\documentclass[
]{article}
\usepackage{amsmath,amssymb}
\usepackage{lmodern}
\usepackage{ifxetex,ifluatex}
\ifnum 0\ifxetex 1\fi\ifluatex 1\fi=0 % if pdftex
  \usepackage[T1]{fontenc}
  \usepackage[utf8]{inputenc}
  \usepackage{textcomp} % provide euro and other symbols
\else % if luatex or xetex
  \usepackage{unicode-math}
  \defaultfontfeatures{Scale=MatchLowercase}
  \defaultfontfeatures[\rmfamily]{Ligatures=TeX,Scale=1}
\fi
% Use upquote if available, for straight quotes in verbatim environments
\IfFileExists{upquote.sty}{\usepackage{upquote}}{}
\IfFileExists{microtype.sty}{% use microtype if available
  \usepackage[]{microtype}
  \UseMicrotypeSet[protrusion]{basicmath} % disable protrusion for tt fonts
}{}
\makeatletter
\@ifundefined{KOMAClassName}{% if non-KOMA class
  \IfFileExists{parskip.sty}{%
    \usepackage{parskip}
  }{% else
    \setlength{\parindent}{0pt}
    \setlength{\parskip}{6pt plus 2pt minus 1pt}}
}{% if KOMA class
  \KOMAoptions{parskip=half}}
\makeatother
\usepackage{xcolor}
\IfFileExists{xurl.sty}{\usepackage{xurl}}{} % add URL line breaks if available
\IfFileExists{bookmark.sty}{\usepackage{bookmark}}{\usepackage{hyperref}}
\hypersetup{
  pdftitle={assignment 1},
  hidelinks,
  pdfcreator={LaTeX via pandoc}}
\urlstyle{same} % disable monospaced font for URLs
\usepackage[margin=1in]{geometry}
\usepackage{graphicx}
\makeatletter
\def\maxwidth{\ifdim\Gin@nat@width>\linewidth\linewidth\else\Gin@nat@width\fi}
\def\maxheight{\ifdim\Gin@nat@height>\textheight\textheight\else\Gin@nat@height\fi}
\makeatother
% Scale images if necessary, so that they will not overflow the page
% margins by default, and it is still possible to overwrite the defaults
% using explicit options in \includegraphics[width, height, ...]{}
\setkeys{Gin}{width=\maxwidth,height=\maxheight,keepaspectratio}
% Set default figure placement to htbp
\makeatletter
\def\fps@figure{htbp}
\makeatother
\setlength{\emergencystretch}{3em} % prevent overfull lines
\providecommand{\tightlist}{%
  \setlength{\itemsep}{0pt}\setlength{\parskip}{0pt}}
\setcounter{secnumdepth}{-\maxdimen} % remove section numbering
\usepackage{fancyhdr}
\pagestyle{fancy}
\fancyhead[CO,CE]{Group 9}
\fancyfoot[CO,CE]{https://github.com/KazuMaeshima/Group-9-.git}
\ifluatex
  \usepackage{selnolig}  % disable illegal ligatures
\fi

\title{assignment 1}
\author{}
\date{\vspace{-2.5em}}

\begin{document}
\maketitle

\hypertarget{load-data}{%
\subsection{Load data}\label{load-data}}

avocado=read.csv(file
=``\url{https://github.com/KazuMaeshima/Group-9-/raw/main/avocado.csv}'',
header =TRUE) \#\#Provide a introduction of your analysis in the .RMD
file so it can be produced in the output \# this codes will introduce us
how to use R Studio as part of our day to day data analysis. It will
produce variables, mean,median,mode, show and manipulate data and plot
graphs using ggplot2 \#\# head str(avocado) \#\# Print the structure of
your dataset. print(avocado) \#\#List the variables in your dataset
names(avocado) \#\#Print the top 15 rows of your dataset.
head(avocado,15) \#\#Write a user defined function using any of the
variables from the data set m \textless- c(45,34,34,34,67)

getmode \textless- function(m) \{ uniqv \textless- unique(m)
uniqv{[}which.max(tabulate(match(m, uniqv))){]} \} getmode(m) \#\#Use
data manipulation techniques and filter rows based on any logical
criteria that exist in your dataset
filter(Avocado,AveragePrice\textless1) \#\#Identify the dependent \&
independent variables and use reshaping techniques and create a new data
frame by joining those variablesfrom your dataset.\\
\# Create a new dataset with the selected columns bags \textless-
as.data.frame(avocado \%\textgreater\%
select(Total.Bags,Small.Bags,Large.Bags,XLarge.Bags)) \#\#Remove missing
values in your dataset. na.omit(Avocado) \#\#Identify and remove
duplicated data in your dataset. avocado{[}!duplicated(avocado), {]}
\#\#Reorder multiple rows in descending order avocado \%\textgreater\%
arrange(desc(AveragePrice)) \#\#Rename some of the column names in your
dataset. head(avocado) m \textless- avocado dim(m) col\_name \textless-
paste(``Col'', 1:14, sep = "") head(m) names(m) \textless-col\_name
head(m)

\#\#Add new variables in your data frame by using a mathematical
function (for e.g.~--multiply an existing column by 2 and add it as a
new variable to your data frame) \#Create new variable by mutliplying an
existing column by 2

avocado\(Doubleyear = avocado\)year*2

\#\#Create a training set using random number generator engine. \#
Initiate random number generator engine

set.seed(1234)

\hypertarget{select-80-rows-from-the-main-dataset-as-the-training-set}{%
\subsection{Select 80\% rows from the main dataset as the training
set}\label{select-80-rows-from-the-main-dataset-as-the-training-set}}

training = avocado \%\textgreater\% sample\_frac(0.8,replace=FALSE)

\#Print the summary statistics of your dataset.

summary(avocado)

\#\#Use any of the numerical variables from the dataset and perform the
following statistical functions. Mean mean(avocado\$Large.Bags)

\#\#Median median(avocado\$Total.Bags)

\#\#Mode v \textless- c(avocado\$AveragePrice) \# Calculate the mode
using the user defined function result \textless- getmode(v)
print(result)

\#\#Range range(avocado\$Total.Bags)

\#\#Plot a scatter plot for any 2 variables in your dataset. ggplot(data
= avocado, aes(x = Total.Bags, y=AveragePrice))+geom\_point()

\#\#Plot a bar plot for any 1 variables in your dataset ggplot(data =
avocado, aes(x = AveragePrice))+geom\_bar()

\#\#Find the correlation between any 2 variables by applying least
square linear regression model. library(ggpubr)

ggscatter(avocado, x=``Total.Bags'', y=``AveragePrice'',
add=``reg.line'', conf.int = TRUE, cor.coef = TRUE, cor.method =
``pearson'', xlab = ``Total Bags'',ylab = ``Average Price'')

\#\#Provide a conclusion of your analysis if any in the .RMD file.
\#there are different types of avocados. Each avocados corresponds with
their respective price based on their size and each country prices their
avocados differently

\end{document}
